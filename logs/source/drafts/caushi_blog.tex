\documentclass{article}

\title{A case for an open digital water systems}
\author{Abhiram Mullapudi, Ph.D.}

\begin{document}
\maketitle
\section{Introduction}
The last decade has seen signigicant focus on the 


Digital transformation of water systems has been in progress for a while.
But there is no wide spread adoptoion. 
There has been a increate in the new works and increasing research in last 10 years. 
Despirte the popularity of research, its adoption in for real cities has been limited. 

In this blog post, I will provide an overview of the what we can leverage the lessoons from other feilds that have undergone digital transformation and provide a way for us to ursher in an era of digital water systems. 

1. There is a lack of exchange between academia and industry. 
2. We dont have industry standards on how we can evaluate and rank digital products
3. Best-practices on how we can create solutions for industry part

\section{Background}
There are a lot of control papers published over the years. But we have very few documented studies on it being applied in the actual systems.

Over the years 
\section{Methodolody}
Reproducability work using docerization 
Open Model MP from Caleb


1. Open Source Infrastrucure
This will help us set to 
2. Eco systems

open-storm 
pyswmm
pystorms
examples from other groups
standards on how we can communicate information 


3. Foundation to help us guide the development


\section{Discussion}
How each of these above can help us address these challenges
\section{Conclusion}
\end{document}
