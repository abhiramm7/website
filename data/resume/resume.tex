\documentclass{article}

\usepackage{hyperref}
\usepackage{geometry}
\usepackage{fontspec}
\usepackage{multicol}
\usepackage{paracol}
\usepackage{xcolor}
\usepackage{titlesec}
\usepackage{fontawesome5}
\usepackage{enumitem}

% Tufte-inspired color palette
\definecolor{tuftegray}{RGB}{102, 102, 102}
\definecolor{tufteaccent}{RGB}{170, 65, 53}
\definecolor{tuftedark}{RGB}{50, 50, 50}

% Page geometry - generous margins in Tufte style
\geometry{left=1.5cm, right=1.5cm,top=1.5cm, bottom=1.5cm, nohead,nofoot}

% Typography
\setmainfont{EB Garamond}[Scale=1.0, BoldFont={* Bold}, ItalicFont={* Italic}]
\setsansfont{Gill Sans}[Scale=0.9, BoldFont={* Bold}]

% Hyperlink styling
\hypersetup{colorlinks=true, urlcolor=tufteaccent,linkcolor=tuftedark,citecolor=tuftedark}

\pagestyle{empty}

\titleformat{\section}
  {\normalfont\large\scshape\color{tuftedark}}
  {}{0em}{}[\titlerule]
\titlespacing*{\section}{0pt}{1.5ex plus 1ex minus .2ex}{1ex plus .2ex}

\titleformat{\subsection}
  {\normalfont\normalsize\itshape\color{tuftegray}}
  {}{0em}{}
\titlespacing*{\subsection}{0pt}{1.25ex plus 1ex minus .2ex}{0.5ex plus .2ex}

\newcommand{\degreeyear}[2]{
  {\textbf{#1}}\\
  {\small\textcolor{tuftegray}{#2}}
}

\newcommand{\divider}{\textcolor{tuftegray}{\rule{\linewidth}{0.5pt}}}

\newcommand{\publication}[3]{
  \textbf{#1} \hfill \textcolor{tuftegray}{#2}\\
  {\small\textit{#3}}\\[0.5em]
}

\newcommand{\experience}[4]{
  \textbf{#1}, \textit{#2} \hfill \textcolor{tuftegray}{#3}\\
  #4
}

\setlength{\columnseprule}{0pt}
\setlength{\columnsep}{1cm}
\columnratio{0.7}
\begin{document}
% remove page numbers
\setlength{\parindent}{0pt}

% main header with contact information
\LARGE\textbf{Abhiram Mullapudi}\\
\vspace{1em}
\footnotesize\faEnvelope\hspace{0.5em}\href{mailto:abhiramm@umich.edu}{\footnotesize abhiramm@umich.edu}\hspace{0.5em}
\footnotesize\faGithub\hspace{0.5em}\href{https://github.com/abhiramm7}{\footnotesize abhiramm7}\hspace{0.5em}
\footnotesize\faGlobe\hspace{0.5em}\href{https://randomstorms.net}{\footnotesize randomstorms.net}

\begin{paracol}{2}

\section{Experience}
\experience{Senior Data Scientist}{Xylem}{2023--Present}{
I design and implement end-to-end machine learning-based solutions that inform decision-making in urban water infrastructure systems.
\begin{itemize}[leftmargin=*,itemsep=1mm]
  \item Leading the development of statistical and machine learning-based methodologies for time-series filtering and anomaly detection for predictive maintenance and operational decision-making.
  \item Currently developing a Flyte-based MLOps platform to streamline end-to-end machine learning model development, deployment, and maintenance for Xylem's digital water products.
\end{itemize}
}
\vspace{0.5em}

\experience{Hydraulic Control and Optimization Engineer}{Xylem}{2020--2023}{
Pioneered advanced machine learning and data engineering solutions for urban water infrastructure, transforming raw sensor data into actionable intelligence that optimizes water network performance.
\begin{itemize}[leftmargin=*,itemsep=1mm]
  \item Developed a 1D-CNN model that leverages NOAA rainfall forecasts and near-real-time flow measurements to accurately predict 24-hour inflow to water treatment plants.
  \item Engineered an advanced 1D-CNN interpolation framework for processing spatially distributed river level data, enabling comprehensive environmental monitoring.
  \item Designed a high-performance real-time processing system leveraging symbolic programming and advanced statistical techniques to detect network irregularities across 600+ concurrent data streams.
\end{itemize}
}

\section{Publications}
\publication{Identification of stormwater control strategies and their associated uncertainties using Bayesian Optimization}{2023}{arXiv preprint}

\publication{pystorms: a simulation sandbox for the design and evaluation of stormwater control algorithms}{2023}{Environmental Modelling and Software}

\publication{Improvement of phosphorus removal in bioretention cells using real-time control}{2022}{Urban Water Journal}

\publication{StormReactor: An open-source Python package for the integrated modeling of urban water quality and water balance}{2021}{Environmental Modelling \& Software}

\publication{Deep Reinforcement Learning for the Real Time Control of Stormwater Systems}{2020}{Advances in Water Resources}


\switchcolumn

% SIDEBAR COLUMN (30%) ===============================
\begin{flushleft}

\section{Education}
\vspace{0.5em}
\degreeyear{Ph.D. in Civil Engineering}{2017--2020}
University of Michigan\\
\vspace{0.2em}
{\small\textit{Statistical Learning Approaches for the Control of Stormwater Systems}}\\
{\small Advisor: Dr. Branko Kerkez}

\vspace{0.8em}
\degreeyear{M.Sc.Eng. in Civil Engineering}{2015--2017}
University of Michigan

\vspace{0.8em}
\degreeyear{B.Tech. in Civil Engineering}{2011--2015}
Amrita Vishwa Vidyapeetham
\vspace{1em}
\divider
\vspace{1em}

\section{Skills}
\vspace{0.5em}
\textbf{Programming}\\
{\small Python, MATLAB, C/C++, \LaTeX, SQL, Bash}
\vspace{0.8em}

\textbf{ML \& Data Science}\\
{\small PyTorch, TensorFlow, JAX, scikit-learn}
\vspace{0.8em}

\textbf{Cloud \& MLOps}\\
{\small MLflow, Flyte, AWS, Google Cloud, Azure}
\vspace{1em}
\divider
\vspace{1em}

\section{Awards}
\vspace{0.5em}
\textbf{LIFT Intelligent Water Systems Challenge}\\
{\small\textcolor{tuftegray}{Grand Prize Winner, 2019}}
\vspace{0.8em}

\textbf{Academic Excellence}\\
{\small\textcolor{tuftegray}{Amrita Vishwa Vidhyapeetham, 2013, 2015}}
\vspace{1em}
\divider
\vspace{1em}

\section{Projects \& Software}
\textbf{pystorms} \hfill \textcolor{tuftegray}{Creator}\\
{\small An open-source Python library for the design and evaluation of stormwater control algorithms.}
\vspace{0.8em}

\textbf{pyswmm} \hfill \textcolor{tuftegray}{Maintainer}\\
{\small The Python interface to Stormwater Management Model (SWMM), the industry standard for modeling stormwater systems.}
\vspace{0.8em}

\textbf{rrcf} \hfill \textcolor{tuftegray}{Contributor}\\
{\small An open-source implementation of the Robust Random Cut Forest algorithm for anomaly detection on streams.}




\end{flushleft}
\end{paracol}
\end{document}
