%!TEX TS-program = lualatex
%!TEX encoding   = UTF-8 Unicode

% This is the simplest class to start with. You may want to change the
% paper format or the standard font size.
\documentclass[a4paper,11pt]{article}

\usepackage{fontspec}    % Font selection
\usepackage{fontawesome} % Symbols/icons

% Change this depending on your own preferences. I personally prefer
% nice ligatures and 'old style' numbers. Additionally, the way each
% font is set up ensures that the layout is very uniform---mostly, I
% achieve this effect using `MatchLowercase`.
%
% TODO: Make sure you select a font with small caps here, unless you
% also change the section style. See below.
\defaultfontfeatures{Ligatures=TeX}
\setmainfont[Renderer=Basic,Numbers={Proportional,OldStyle}]{Minion Pro}
\setsansfont[Scale=MatchLowercase]{Cabin Regular}
\setmonofont[Scale=MatchLowercase]{Fira Mono}

\setlength{\marginparsep}{10pt} % Separation of margin notes

%%%%%%%%%%%%%%%%%%%%%%%%%%%%%%%%%%%%%%%%%%%%%%%%%%%%%%%%%%%%%%%%%%%%%%%%
% Document layout
%%%%%%%%%%%%%%%%%%%%%%%%%%%%%%%%%%%%%%%%%%%%%%%%%%%%%%%%%%%%%%%%%%%%%%%%

% Permits a very precise adjustment of all the margins. Since a CV is
% a document where *you* should exhibit *your* preferences, feel free
% to adjust this the way you want.
%
% I am not aware of any good rules here.
\usepackage{geometry}
\geometry{a4paper, textwidth=14.0cm,textheight=25.0cm,marginparwidth=2.5cm}

\setlength{\parindent}{0pt}     % No indentations for paragraphs
\setlength{\skip\footins}{2cm}  % Reduced footer distance

% A nice way to typeset proper ordinal superscripts
\newcommand{\rd}{\textsuperscript{\textup{rd}}\xspace}
\newcommand{\nd}{\textsuperscript{\textup{nd}}\xspace}
\renewcommand{\th}{\textsuperscript{\textup{th}}\xspace}

% To be used to indicate equal contributions of two or more authors in
% a paper.
\newcommand{\authorequal}{\kern-0.1em\textsuperscript{\dagger}}

%%%%%%%%%%%%%%%%%%%%%%%%%%%%%%%%%%%%%%%%%%%%%%%%%%%%%%%%%%%%%%%%%%%%%%%%
% Margins macro
%%%%%%%%%%%%%%%%%%%%%%%%%%%%%%%%%%%%%%%%%%%%%%%%%%%%%%%%%%%%%%%%%%%%%%%%
%
% This permits you to put some comments into the margins of the
% document; I borrowed this idea from Tufte's works, but in the
% CV, I only use it to indicate durations.

\usepackage{marginnote}
\renewcommand*{\raggedleftmarginnote}{}
\newcommand{\years}[1]{%
  % Forces notes to appear on the left
  {\reversemarginpar\strut\marginnote{{\small#1}}}%
}


% Required to support footnote references; has to come *after* the
% `marginnote` package, though, because it redefines some things.
\usepackage{scrextend}

%%%%%%%%%%%%%%%%%%%%%%%%%%%%%%%%%%%%%%%%%%%%%%%%%%%%%%%%%%%%%%%%%%%%%%%%
% Section style
%%%%%%%%%%%%%%%%%%%%%%%%%%%%%%%%%%%%%%%%%%%%%%%%%%%%%%%%%%%%%%%%%%%%%%%%

% This styles *all* the sections exactly the same: regular font, with
% small caps.
%
% TODO: Adjust this if you change the font---otherwise, everything will
% look super strange.
\usepackage{sectsty}
\allsectionsfont{\mdseries\scshape}

% Typographical adjustments. If you have longer blocks of texts, such as
% publications, in your CV, this will give the text a more natural look.
%
% TODO: Some people do not like this. If you are among them, just change
% this or remove it
\usepackage{microtype}

%%%%%%%%%%%%%%%%%%%%%%%%%%%%%%%%%%%%%%%%%%%%%%%%%%%%%%%%%%%%%%%%%%%%%%%%
% Colours
%%%%%%%%%%%%%%%%%%%%%%%%%%%%%%%%%%%%%%%%%%%%%%%%%%%%%%%%%%%%%%%%%%%%%%%%

% TODO: redefine this colour if you do not like it.
\usepackage[usenames,dvipsnames]{color}
\definecolor{cardinal}{RGB}{196, 30, 58}

%%%%%%%%%%%%%%%%%%%%%%%%%%%%%%%%%%%%%%%%%%%%%%%%%%%%%%%%%%%%%%%%%%%%%%%%
% PDF setup
%%%%%%%%%%%%%%%%%%%%%%%%%%%%%%%%%%%%%%%%%%%%%%%%%%%%%%%%%%%%%%%%%%%%%%%%

\usepackage[
  bookmarks,  % Create PDF bookmarks
  colorlinks, % Use the link colour from above
  breaklinks  % Break links in the text
]{hyperref}

\hypersetup{%
  urlcolor=cardinal,
  pdfauthor={Abhiram Mullapudi},                  
  pdftitle={Abhiram Mullapudi: Curriculum vitae},
  pdfproducer={}
}

%%%%%%%%%%%%%%%%%%%%%%%%%%%%%%%%%%%%%%%%%%%%%%%%%%%%%%%%%%%%%%%%%%%%%%%%
% Penalties
%%%%%%%%%%%%%%%%%%%%%%%%%%%%%%%%%%%%%%%%%%%%%%%%%%%%%%%%%%%%%%%%%%%%%%%%

\clubpenalty=10000
\displaywidowpenalty=10000
\widowpenalty=10000

\usepackage[defaultlines=100,all]{nowidow} % rejects all orphaned and widowed lines
\setlength\parindent{0pt}
\usepackage{vwcol}
%%%%%%%%%%%%%%%%%%%%%%%%%%%%%%%%%%%%%%%%%%%%%%%%%%%%%%%%%%%%%%%%%%%%%%%%
% Main document
%%%%%%%%%%%%%%%%%%%%%%%%%%%%%%%%%%%%%%%%%%%%%%%%%%%%%%%%%%%%%%%%%%%%%%%%
\begin{document}

{\huge \textsc{Abhiram Mullapudi}}\\
{\small \textit{Curriculum vitae} (\today)}\\

\begin{vwcol}[widths={0.78,0.22}]
{\footnotesize Cyber-physical systems are a promising area of exploration for developing resilient urban water
systems in the face of extreme weather events. I am interested in addressing knowledge gaps and building technologies to
create a new generation of robust, equitable, and sustainable cyber-physical water infrastructure.}

{\scriptsize {\noindent {\faEnvelope\space{\href{mailto:abhiramm@umich.edu}{abhiramm@umich.edu}}}\\
{\faHome\space{\href{https://randomstorms.net}{randomstorms.net}}}\\
{\faGithub\space{\href{https://github.com/abhiramm7}{abhiramm7}}}\\}}
\end{vwcol}

\section*{Positions}

\years{2023--}
\textbf{Senior Data Scientist} at Xylem\\

I design and implement machine learning-based solutions that inform decision-making in urban water infrastructure systems.

\vspace{2mm}
{\small \textbf{Key Highlights:}}
\vspace{-2mm}
\begin{itemize}
	\setlength\itemsep{1mm}
	\item  Spearheadeding development of statistical and AI-powered methodologies for
time-series filtering and anomaly detection, enabling predictive maintenance
strategies in water networks.
	\item  Currently developing a Flyte-based MLOps platform to streamline
end-to-end machine learning model development, deployment, and maintenance for
Xylem's digital water products.
\end{itemize}

\years{2020--2023}
\textbf{Hydraulic Control and Optimization Engineer} at Xylem\\

Spearheaded the design and implementation of cutting-edge digital water solutions
that drive predictive maintenance, operational efficiency, and informed
decision-making in cities and utilities worldwide.

\vspace{2mm}
{\small \textbf{Key Highlights:}}
\vspace{-2mm}
\begin{itemize}
	\setlength\itemsep{1mm}
	\item \textbf{Real-Time Flow Rate Prediction}: Developed a 1D-CNN model that
leverages NOAA rainfall forecasts and near-real-time flow measurements to accurately predict 24-hour inflow to water treatment plants.
	\item \textbf{River Level Prediction}: Created an 1D-CNN-based model that
filters and interpolates spatially-temporally distributed river levels for
reporting to regulatory organizations.
	\item \textbf{Real-Time Data Processing}: Designed a high-performance real-time time series processing module that utilizes symbolic programming and statistical methodologies
to detect anomalies in water networks handling over 600 data streams.
	\item \textbf{Inflow Prediction and CSO Event Detection}: Developed an AI-powered dashboard that predicts inflows to treatment plants, enabling informed decision-making, and a
real-time dashboard identifying Combined Sewer Overflow (CSO) events.
\end{itemize}

\vspace{1mm}
{\small \textbf{Supportive Functions:}}
\vspace{-2mm}
\begin{itemize}
	\setlength\itemsep{1mm}
	\item Maintained and updated critical real-time services, ETL scripts, and internal
databases to ensure seamless operation of Xylem Vue's Waste Water Network Optimization Solution.
\item Collaborated with cross-functional teams to integrate digital water solutions
into existing infrastructure, driving successful adoption and maximized benefits
for clients.
\end{itemize}

%%%%%%%%%%%%%%%%%%%%%%%%%%%%%%%%%%%%%%%%%%%%%%%%%%%%%%%%%%%%%%%%%%%%%%%%
\section*{Education}
%%%%%%%%%%%%%%%%%%%%%%%%%%%%%%%%%%%%%%%%%%%%%%%%%%%%%%%%%%%%%%%%%%%%%%%%

\years{2017--2020}%
\textbf{Ph.D.}\ in Civil Engineering at University of Michigan, Ann Arbor, USA\\[.1cm]
\noindent \emph{Statistical Learning Approaches for the Control of Stormwater Systems}\\
Advisor: Dr.\ Branko Kerkez\\[.1cm]

\years{2015--2017}%
\textbf{M.Sc.Eng.}\ in Civil Engineering at University of Michigan, Ann Arbor, USA\\[.1cm]

\years{2011--2015}%
\textbf{B.Tech}\ in Civil Engineering at Amrita Vishwa Vidyapeetham, Coimbatore, India

%%%%%%%%%%%%%%%%%%%%%%%%%%%%%%%%%%%%%%%%%%%%%%%%%%%%%%%%%%%%%%%%%%%%%%%%
\section*{Publications}
%%%%%%%%%%%%%%%%%%%%%%%%%%%%%%%%%%%%%%%%%%%%%%%%%%%%%%%%%%%%%%%%%%%%%%%%
\years{2023} \textbf{Abhiram Mullapudi} and Branko Kerkez.\ \href{https://arxiv.org/abs/2305.18630}{\emph{Identification of stormwater control strategies and their associated uncertainties using Bayesian Optimization.}}\\[.1cm]

\years{2023} Sara P. Rimer, \textbf{Abhiram Mullapudi}, Sara C. Troutman, Gregory Ewing, Jeffrey M. Sadler, Jonathan L. Goodall, Ruben Kertesz, Jon M. Hathaway, and Branko Kerkez.\ \href{https://www.sciencedirect.com/science/article/abs/pii/S136481522300021X}{\emph{pystorms: a simulation sandbox for the design and evaluation of stormwater control algorithms.}} Environmental Modelling and Software, 2023\\[.1cm]

\years{2022} Brooke E. Mason, \textbf{Abhiram Mullapudi}, Cyndee Gruden, and Branko Kerkez.\ \href{https://www.tandfonline.com/doi/abs/10.1080/1573062X.2022.2108464}{\emph{Improvement of phosphorus removal in bioretention cells using real-time control}}. Urban Water Journal, 19(9):992–998, 2022\\[.1cm]


\years{2021} Brooke E. Mason, \textbf{Abhiram Mullapudi}, and Branko Kerkez.\ \href{https://www.sciencedirect.com/science/article/abs/pii/S1364815221002176}{\emph{StormReactor: An open-source Python package for the integrated modeling of urban water quality and water balance}}. Environmental Modelling \& Software, 145:105175, 2021\\[.1cm]


\years{2020} \textbf{Abhiram Mullapudi}.\ \href{https://github.com/abhiramm7/UofM-Doc-Thesis/blob/master/Dissertation_Abhi.pdf}{\emph{Statistical Learning Approaches For The Control Of Stormwater Systems}}. PhD thesis, University of Michigan, Ann Arbor, 2020\\[.1cm]


\years{2020}  Bryant E. McDonnell, Katherine Ratliff, Michael E. Tryby, Jennifer Jia Xin Wu,
and \textbf{Abhiram Mullapudi}.\ \href{https://joss.theoj.org/papers/10.21105/joss.02292.pdf}{\emph{PySWMM: The Python Interface to Stormwater Management Model (SWMM)}}. Journal of Open Source Software, 5(52):2292, 2020\\[.1cm]


\years{2020}  \textbf{Abhiram Mullapudi}, Matthew Lewis, Cyndee Gruden, and Branko Kerkez.
\href{http://www.sciencedirect.com/science/article/pii/S0309170820302499}{\emph{Deep Reinforcement Learning for the Real Time Control of Stormwater Systems}}. Advances in Water Resources, 2020\\[.1cm]


\years{2019} Matthew D. Bartos, \textbf{Abhiram Mullapudi}, and Sara C. Troutman.\ \href{https://joss.theoj.org/papers/10.21105/joss.01336}{\emph{rrcf: Implementation of the Robust Random Cut Forest algorithm for anomaly detection on streams}}. The Journal of Open Source Software, 4:1336, 2019\\[.1cm]


\years{2018} \textbf{Abhiram Mullapudi}, Matthew D. Bartos, Brandon P. Wong, and Branko Kerkez.
\href{https://randomstorms.net/data/papers/shapingwater.pdf}{\emph{Shaping Streamflow Using a Real-Time Stormwater Control Network}}. Sensors, 18(7):2259, Jul 2018\\[.1cm]


\years{2017} \textbf{Abhiram Mullapudi}, Brandon P. Wong, and Branko Kerkez.\ \href{https://randomstorms.net/data/papers/stormwatertheory.pdf}{\emph{Emerging investigators series: building a theory for smart stormwater systems}}. Environmental Science: Water Research \& Technology, 3(1):66–77, 2017


%%%%%%%%%%%%%%%%%%%%%%%%%%%%%%%%%%%%%%%%%%%%%%%%%%%%%%%%%%%%%%%%%%%%%%%%
\section*{WORKSHOPS AND SPECIAL SESSIONS}
%%%%%%%%%%%%%%%%%%%%%%%%%%%%%%%%%%%%%%%%%%%%%%%%%%%%%%%%%%%%%%%%%%%%%%%%

\years{2023} \emph{Technical Workshop: Building the Next Generation of Intelligent Urban Water
Systems: A Hands-on Workshop on Digital Twin-based Solutions}\\[.05cm]

Organized and led a workshop session at the ASCE’s EWRI conference on building
digital water systems.

\years{2022} \emph{Moving towards an open urban water modeling paradigm: perspectives from
academia and industry}\\[.05cm]

Organized a special session at the Urban Drainage Modeling conference on the
role of open-source software in ushering the era of smart urban water systems.

\years{2022} \emph{UDS-RTC 101: A hands-on workshop on the real-time control of the urban
drainage systems}\\[.05cm]

Organized and led a pre-conference workshop at the Urban Drainage Modeling
conference on the control of stormwater systems attended by an international
group of researchers and practitioners.

\years{2017, 2019} \emph{CUAHSI Open Source Urban Hydrology Sensor Bootcamp}\\[.05cm]

Co-organized and led a three day workshop on the use open-storm’s sensing stack
for the monitoring and control of stormwater systems.

\section*{Skills}

\section*{Professional Service}
\begin{itemize}
	\item Vice-chair of Emerging and Innovative Technologies subcommittee for American Socity of Civil Engneering's Environmental and Water Research Congress. 
	\item Part of the organizing committee at 2024 NURAL-IPS Gaussian Processes workshop
\end{itemize}
\subsection*{Journal Review}
\begin{itemize}
	\item \textit{HardwareX}
	\item \textit{IEEE-CDC 2020}
	\item \textit{Journal of Hydrology}
	\item \textit{Water Resources Research}
	\item \textit{Journal of Hydroinformatics}
	\item \textit{Water Science and Technology}
	\item \textit{Journal of Open Source Software}
	\item \textit{Journal of Computing in Civil Engineering}
	\item \textit{Journal of Irrigation and Drainage Engineering}
	\item \textit{Journal of Water Resources Planning and Management}
	\item \textit{Environmental Science: Water Research \& Technology}
\end{itemize}

\section*{Conference}


%%%%%%%%%%%%%%%%%%%%%%%%%%%%%%%%%%%%%%%%%%%%%%%%%%%%%%%%%%%%%%%%%%%%%%%%
\section*{About this template}
%%%%%%%%%%%%%%%%%%%%%%%%%%%%%%%%%%%%%%%%%%%%%%%%%%%%%%%%%%%%%%%%%%%%%%%%

This document is modelled after the style of
\href{https://bastian.rieck.me}{my} own CV, which you can find
\href{https://bastian.rieck.me/about/cv.pdf}{here}. It was originally
inspired by the CV of \href{http://nitens.org/taraborelli/cvtex}{Dario
Taraborelli}, but has since somewhat evolved---or so I like to tell
myself---from the original template.\\

This document does not offer many special features, except for the
\verb|\years| macro, which can be used to typeset small notes in the
margin of the document. I use them to indicate \emph{durations}, but you
could also repurpose it to have small annotations in the style of Edward
Tufte. In addition, the template uses two special fonts for sans-serif
typesetting and monospace typesetting. I do not tend to use the former
for many things, but you might like it for typesetting the titles of
publications. The latter font type, though, I often use in order to
describe software projects or packages, such as \texttt{PyTorch} or
\texttt{scikit-learn}.\\

That's all there is to it---enjoy the template \& feel free to open
tickets for comments or feedback.

\end{document}
