%!TEX TS-program = lualatex
%!TEX encoding   = UTF-8 Unicode

% This is the simplest class to start with. You may want to change the
% paper format or the standard font size.
\documentclass[a4paper,11pt]{article}

\usepackage{fontspec}    % Font selection
\usepackage{fontawesome} % Symbols/icons

% Change this depending on your own preferences. I personally prefer
% nice ligatures and 'old style' numbers. Additionally, the way each
% font is set up ensures that the layout is very uniform---mostly, I
% achieve this effect using `MatchLowercase`.
%
% TODO: Make sure you select a font with small caps here, unless you
% also change the section style. See below.
\defaultfontfeatures{Ligatures=TeX}
\setmainfont[Renderer=Basic,Numbers={Proportional,OldStyle}]{Minion Pro}
\setsansfont[Scale=MatchLowercase]{Cabin Regular}
\setmonofont[Scale=MatchLowercase]{Fira Mono}

\setlength{\marginparsep}{10pt} % Separation of margin notes

%%%%%%%%%%%%%%%%%%%%%%%%%%%%%%%%%%%%%%%%%%%%%%%%%%%%%%%%%%%%%%%%%%%%%%%%
% Document layout
%%%%%%%%%%%%%%%%%%%%%%%%%%%%%%%%%%%%%%%%%%%%%%%%%%%%%%%%%%%%%%%%%%%%%%%%

% Permits a very precise adjustment of all the margins. Since a CV is
% a document where *you* should exhibit *your* preferences, feel free
% to adjust this the way you want.
%
% I am not aware of any good rules here.
\usepackage{geometry}
\geometry{a4paper, textwidth=14.0cm,textheight=25.0cm,marginparwidth=2.5cm}

\setlength{\parindent}{0pt}     % No indentations for paragraphs
\setlength{\skip\footins}{2cm}  % Reduced footer distance

% A nice way to typeset proper ordinal superscripts
\newcommand{\rd}{\textsuperscript{\textup{rd}}\xspace}
\newcommand{\nd}{\textsuperscript{\textup{nd}}\xspace}
\renewcommand{\th}{\textsuperscript{\textup{th}}\xspace}

% To be used to indicate equal contributions of two or more authors in
% a paper.
\newcommand{\authorequal}{\kern-0.1em\textsuperscript{\dagger}}

%%%%%%%%%%%%%%%%%%%%%%%%%%%%%%%%%%%%%%%%%%%%%%%%%%%%%%%%%%%%%%%%%%%%%%%%
% Margins macro
%%%%%%%%%%%%%%%%%%%%%%%%%%%%%%%%%%%%%%%%%%%%%%%%%%%%%%%%%%%%%%%%%%%%%%%%
%
% This permits you to put some comments into the margins of the
% document; I borrowed this idea from Tufte's works, but in the
% CV, I only use it to indicate durations.

\usepackage{marginnote}
\renewcommand*{\raggedleftmarginnote}{}
\newcommand{\years}[1]{%
  % Forces notes to appear on the left
  {\reversemarginpar\strut\marginnote{{\small#1}}}%
}


% Required to support footnote references; has to come *after* the
% `marginnote` package, though, because it redefines some things.
\usepackage{scrextend}

%%%%%%%%%%%%%%%%%%%%%%%%%%%%%%%%%%%%%%%%%%%%%%%%%%%%%%%%%%%%%%%%%%%%%%%%
% Section style
%%%%%%%%%%%%%%%%%%%%%%%%%%%%%%%%%%%%%%%%%%%%%%%%%%%%%%%%%%%%%%%%%%%%%%%%

% This styles *all* the sections exactly the same: regular font, with
% small caps.
%
% TODO: Adjust this if you change the font---otherwise, everything will
% look super strange.
\usepackage{sectsty}
\allsectionsfont{\mdseries\scshape}

% Typographical adjustments. If you have longer blocks of texts, such as
% publications, in your CV, this will give the text a more natural look.
%
% TODO: Some people do not like this. If you are among them, just change
% this or remove it
\usepackage{microtype}

%%%%%%%%%%%%%%%%%%%%%%%%%%%%%%%%%%%%%%%%%%%%%%%%%%%%%%%%%%%%%%%%%%%%%%%%
% Colours
%%%%%%%%%%%%%%%%%%%%%%%%%%%%%%%%%%%%%%%%%%%%%%%%%%%%%%%%%%%%%%%%%%%%%%%%

% TODO: redefine this colour if you do not like it.
\usepackage[usenames,dvipsnames]{color}
\definecolor{cardinal}{RGB}{196, 30, 58}

%%%%%%%%%%%%%%%%%%%%%%%%%%%%%%%%%%%%%%%%%%%%%%%%%%%%%%%%%%%%%%%%%%%%%%%%
% PDF setup
%%%%%%%%%%%%%%%%%%%%%%%%%%%%%%%%%%%%%%%%%%%%%%%%%%%%%%%%%%%%%%%%%%%%%%%%

\usepackage[
  bookmarks,  % Create PDF bookmarks
  colorlinks, % Use the link colour from above
  breaklinks  % Break links in the text
]{hyperref}

\hypersetup{%
  urlcolor=cardinal,
  pdfauthor={Abhiram Mullapudi},                  
  pdftitle={Abhiram Mullapudi: Curriculum vitae},
  pdfproducer={}
}

%%%%%%%%%%%%%%%%%%%%%%%%%%%%%%%%%%%%%%%%%%%%%%%%%%%%%%%%%%%%%%%%%%%%%%%%
% Penalties
%%%%%%%%%%%%%%%%%%%%%%%%%%%%%%%%%%%%%%%%%%%%%%%%%%%%%%%%%%%%%%%%%%%%%%%%

\clubpenalty=10000
\displaywidowpenalty=10000
\widowpenalty=10000

\usepackage[defaultlines=100,all]{nowidow} % rejects all orphaned and widowed lines
\setlength\parindent{0pt}

%%%%%%%%%%%%%%%%%%%%%%%%%%%%%%%%%%%%%%%%%%%%%%%%%%%%%%%%%%%%%%%%%%%%%%%%
% Main document
%%%%%%%%%%%%%%%%%%%%%%%%%%%%%%%%%%%%%%%%%%%%%%%%%%%%%%%%%%%%%%%%%%%%%%%%
\begin{document}

{\huge \textsc{Abhiram Mullapudi}}\\
{\small \textit{Curriculum vitae} (\today)}\\

Cyber-physical water infrastructure systems are a promising area of exploration for developing resilient urban water systems in the face of extreme weather
events. I am interested in addressing knowledge gaps and building technologies that bridge the gap between digital and physical worlds, creating a new
generation of robust, equitable, and sustainable urban water infrastructure systems.

{\scriptsize
\faEnvelope\space{\href{mailto:abhiramm@umich.edu}{abhiramm@umich.edu \textsuperscript{\faExternalLink}}}
\faHome\space{\href{https://randomstorms.net}{randomstorms.net \textsuperscript{\faExternalLink}}}
\faGithub\space{\href{https://github.com/abhiramm7}{github/abhiramm7 \textsuperscript{\faExternalLink}}}
\faLeanpub\space{\href{https://scholar.google.com/citations?user=W9mUihUAAAAJ&hl=en}{google-scholor/abhiramm \textsuperscript{\faExternalLink}}}}


\section*{Positions}

\years{2023--}
\textbf{Senior Data Scientist} at Xylem
\textbf{Developing Scalable Machine-Learning Solutions for Urban Water Infrastructure}

I am currently leading the development of scalable machine learning-based analytical backend services to support urban water infrastructure. Key
responsibilities include:
\begin{itemize}
    \item
    \textbf{Time Series Forecasting \& Anomaly Detection}:
    Developing and deploying algorithms that leverage both data-driven and classical signal processing methodologies to filter timeseries data from sensors,
detect anomalies, and forecast future trends.
        \begin{itemize}
            \item Utilizing advanced statistical techniques to identify patterns and outliers in sensor data
            \item Implementing machine learning models to predict future water usage and demand
        \end{itemize}
    \item
    \textbf{MLOps}:
    Collaborating on the development of a Flyte-based MLOps platform to streamline machine learning model development, deployment, and maintenance.
        \begin{itemize}
            \item Designing and implementing efficient workflows for model training, validation, and deployment
            \item Ensuring seamless integration with existing data systems and infrastructure
        \end{itemize}
\end{itemize}

\years{2020--2023}
\textbf{Hydraulic Control and Optimization Engineer} at Xylem
As a Hydraulic Control and Optimization Engineer at Xylem, I design and implement advanced hydraulic modeling and optimization solutions to improve water
infrastructure performance. Key responsibilities include:
\begin{itemize}
    \item
    \textbf{Time Series Forecasting and Analysis}:
    Developed a state-of-the-art one-dimensional convolutional neural network (CNN) for predicting 24-hour flow rates at the treatment plant in near-real-time,
enabling enhanced water management decisions.
    \item
    \textbf{Time Series Filtering and Interpolation}:
    Created a sophisticated one-dimensional CNN model to filter and interpolate spatially-temporally correlated time series data, ensuring accurate
representation of complex hydraulic systems.
    \item
    \textbf{Stormwater System Simulation}:
    Extended the EPA-SWMM model by developing an extension for identifying travel times and further integrating sensor data into simulations through dynamic
updates of SWMM files, enhancing the accuracy of stormwater system modeling.
    \item
    \textbf{Real-Time Applications}:
        \begin{itemize}
            \item Developed a real-time time series processing module utilizing symbolic programming to filter sensor data streams, detect anomalies in water
networks handling over 600 data streams in real-time.
                \begin{itemize}
                    \item Applied advanced mathematical equations for anomaly detection and signal processing.
                    \item Developed a GUI-based application for monitoring real-time sensor data and system performance.
                \end{itemize}
            \item Designed an AI-powered dashboard to predict inflows to treatment plants, providing actionable insights for operational management decisions.
        \end{itemize}
    \item
    \textbf{Public Awareness and Education}:
    Created a real-time dashboard to identify Combined Sewer Overflow (CSO) events, enabling the public to safely engage in recreational activities on rivers
while ensuring environmental protection.
\end{itemize}

%%%%%%%%%%%%%%%%%%%%%%%%%%%%%%%%%%%%%%%%%%%%%%%%%%%%%%%%%%%%%%%%%%%%%%%%
\section*{Education}
%%%%%%%%%%%%%%%%%%%%%%%%%%%%%%%%%%%%%%%%%%%%%%%%%%%%%%%%%%%%%%%%%%%%%%%%

\years{2017--2020}%
\textbf{Ph.D.}\ in Civil Engineering at University of Michigan, Ann Arbor, USA\\
\emph{Statistical Learning Approaches for the Control of Stormwater Systems}; Advisor: Branko Kerkez \\[.1cm]

\years{2015--2017}%
\textbf{M.Sc.Eng.}\ in Civil Engineering at University of Michigan, Ann Arbor, USA\\[.1cm]

\years{2011--2015}%
\textbf{B.Tech}\ in Civil Engineering at Amrita Vishwa Vidyapeetham, Coimbatore, India\\[.1cm]

%%%%%%%%%%%%%%%%%%%%%%%%%%%%%%%%%%%%%%%%%%%%%%%%%%%%%%%%%%%%%%%%%%%%%%%%
\section*{Publications}
%%%%%%%%%%%%%%%%%%%%%%%%%%%%%%%%%%%%%%%%%%%%%%%%%%%%%%%%%%%%%%%%%%%%%%%%
\years{2023} Abhiram Mullapudi and Branko Kerkez. Bayesian optimization for shaping
the response of stromwater networks.

\years{2023} Sara P. Rimer, Abhiram Mullapudi, Sara C. Troutman, Gregory Ewing, Jef-
frey M. Sadler, Jonathan L. Goodall, Ruben Kertesz, Jon M. Hathaway, and
Branko Kerkez. pystorms: a simulation sandbox for the design and evaluation
of stormwater control algorithms. Environmental Modelling and Software,
2023

%%%%%%%%%%%%%%%%%%%%%%%%%%%%%%%%%%%%%%%%%%%%%%%%%%%%%%%%%%%%%%%%%%%%%%%%
\section*{Honours}
%%%%%%%%%%%%%%%%%%%%%%%%%%%%%%%%%%%%%%%%%%%%%%%%%%%%%%%%%%%%%%%%%%%%%%%%

\years{2018}
Winner of LIFT challange 

\section*{Skills}

\section*{Professional Service}
\begin{itemize}
	\item Vice-chair of Emerging and Innovative Technologies subcommittee for American Socity of Civil Engneering's Environmental and Water Research Congress. 
	\item Part of the organizing committee at 2024 NURAL-IPS Gaussian Processes workshop
\end{itemize}
\subsection*{Journal Review}
\begin{itemize}
	\item \textit{HardwareX}
	\item \textit{IEEE-CDC 2020}
	\item \textit{Journal of Hydrology}
	\item \textit{Water Resources Research}
	\item \textit{Journal of Hydroinformatics}
	\item \textit{Water Science and Technology}
	\item \textit{Journal of Open Source Software}
	\item \textit{Journal of Computing in Civil Engineering}
	\item \textit{Journal of Irrigation and Drainage Engineering}
	\item \textit{Journal of Water Resources Planning and Management}
	\item \textit{Environmental Science: Water Research \& Technology}
\end{itemize}

\section*{Conference}


%%%%%%%%%%%%%%%%%%%%%%%%%%%%%%%%%%%%%%%%%%%%%%%%%%%%%%%%%%%%%%%%%%%%%%%%
\section*{About this template}
%%%%%%%%%%%%%%%%%%%%%%%%%%%%%%%%%%%%%%%%%%%%%%%%%%%%%%%%%%%%%%%%%%%%%%%%

This document is modelled after the style of
\href{https://bastian.rieck.me}{my} own CV, which you can find
\href{https://bastian.rieck.me/about/cv.pdf}{here}. It was originally
inspired by the CV of \href{http://nitens.org/taraborelli/cvtex}{Dario
Taraborelli}, but has since somewhat evolved---or so I like to tell
myself---from the original template.\\

This document does not offer many special features, except for the
\verb|\years| macro, which can be used to typeset small notes in the
margin of the document. I use them to indicate \emph{durations}, but you
could also repurpose it to have small annotations in the style of Edward
Tufte. In addition, the template uses two special fonts for sans-serif
typesetting and monospace typesetting. I do not tend to use the former
for many things, but you might like it for typesetting the titles of
publications. The latter font type, though, I often use in order to
describe software projects or packages, such as \texttt{PyTorch} or
\texttt{scikit-learn}.\\

That's all there is to it---enjoy the template \& feel free to open
tickets for comments or feedback.

\end{document}
