%!TEX TS-program = lualatex
%!TEX encoding   = UTF-8 Unicode

% This is the simplest class to start with. You may want to change the
% paper format or the standard font size.
\documentclass[a4paper,11pt]{article}

\usepackage{fontspec}    % Font selection
\usepackage{fontawesome} % Symbols/icons

% Change this depending on your own preferences. I personally prefer
% nice ligatures and 'old style' numbers. Additionally, the way each
% font is set up ensures that the layout is very uniform---mostly, I
% achieve this effect using `MatchLowercase`.
%
% TODO: Make sure you select a font with small caps here, unless you
% also change the section style. See below.
\defaultfontfeatures{Ligatures=TeX}
\setmainfont[Renderer=Basic,Numbers={Proportional,OldStyle}]{Minion Pro}
\setsansfont[Scale=MatchLowercase]{Cabin Regular}
\setmonofont[Scale=MatchLowercase]{Fira Mono}

\setlength{\marginparsep}{10pt} % Separation of margin notes

%%%%%%%%%%%%%%%%%%%%%%%%%%%%%%%%%%%%%%%%%%%%%%%%%%%%%%%%%%%%%%%%%%%%%%%%
% Document layout
%%%%%%%%%%%%%%%%%%%%%%%%%%%%%%%%%%%%%%%%%%%%%%%%%%%%%%%%%%%%%%%%%%%%%%%%

% Permits a very precise adjustment of all the margins. Since a CV is
% a document where *you* should exhibit *your* preferences, feel free
% to adjust this the way you want.
%
% I am not aware of any good rules here.
\usepackage{geometry}
\geometry{a4paper, textwidth=14.0cm,textheight=25.0cm,marginparwidth=2.5cm}

\setlength{\parindent}{0pt}     % No indentations for paragraphs
\setlength{\skip\footins}{2cm}  % Reduced footer distance

% A nice way to typeset proper ordinal superscripts
\newcommand{\rd}{\textsuperscript{\textup{rd}}\xspace}
\newcommand{\nd}{\textsuperscript{\textup{nd}}\xspace}
\renewcommand{\th}{\textsuperscript{\textup{th}}\xspace}

% To be used to indicate equal contributions of two or more authors in
% a paper.
\newcommand{\authorequal}{\kern-0.1em\textsuperscript{\dagger}}

%%%%%%%%%%%%%%%%%%%%%%%%%%%%%%%%%%%%%%%%%%%%%%%%%%%%%%%%%%%%%%%%%%%%%%%%
% Margins macro
%%%%%%%%%%%%%%%%%%%%%%%%%%%%%%%%%%%%%%%%%%%%%%%%%%%%%%%%%%%%%%%%%%%%%%%%
%
% This permits you to put some comments into the margins of the
% document; I borrowed this idea from Tufte's works, but in the
% CV, I only use it to indicate durations.

\usepackage{marginnote}
\renewcommand*{\raggedleftmarginnote}{}
\newcommand{\years}[1]{%
  % Forces notes to appear on the left
  {\reversemarginpar\strut\marginnote{{\small#1}}}%
}


% Required to support footnote references; has to come *after* the
% `marginnote` package, though, because it redefines some things.
\usepackage{scrextend}

%%%%%%%%%%%%%%%%%%%%%%%%%%%%%%%%%%%%%%%%%%%%%%%%%%%%%%%%%%%%%%%%%%%%%%%%
% Section style
%%%%%%%%%%%%%%%%%%%%%%%%%%%%%%%%%%%%%%%%%%%%%%%%%%%%%%%%%%%%%%%%%%%%%%%%

% This styles *all* the sections exactly the same: regular font, with
% small caps.
%
% TODO: Adjust this if you change the font---otherwise, everything will
% look super strange.
\usepackage{sectsty}
\allsectionsfont{\mdseries\scshape}

% Typographical adjustments. If you have longer blocks of texts, such as
% publications, in your CV, this will give the text a more natural look.
%
% TODO: Some people do not like this. If you are among them, just change
% this or remove it
\usepackage{microtype}

%%%%%%%%%%%%%%%%%%%%%%%%%%%%%%%%%%%%%%%%%%%%%%%%%%%%%%%%%%%%%%%%%%%%%%%%
% Colours
%%%%%%%%%%%%%%%%%%%%%%%%%%%%%%%%%%%%%%%%%%%%%%%%%%%%%%%%%%%%%%%%%%%%%%%%

% TODO: redefine this colour if you do not like it.
\usepackage[usenames,dvipsnames]{color}
\definecolor{cardinal}{RGB}{196, 30, 58}

%%%%%%%%%%%%%%%%%%%%%%%%%%%%%%%%%%%%%%%%%%%%%%%%%%%%%%%%%%%%%%%%%%%%%%%%
% PDF setup
%%%%%%%%%%%%%%%%%%%%%%%%%%%%%%%%%%%%%%%%%%%%%%%%%%%%%%%%%%%%%%%%%%%%%%%%

\usepackage[
  bookmarks,  % Create PDF bookmarks
  colorlinks, % Use the link colour from above
  breaklinks  % Break links in the text
]{hyperref}

\hypersetup{%
  urlcolor=cardinal,
  pdfauthor={Abhiram Mullapudi},                  
  pdftitle={Abhiram Mullapudi: Curriculum vitae},
  pdfproducer={}
}

%%%%%%%%%%%%%%%%%%%%%%%%%%%%%%%%%%%%%%%%%%%%%%%%%%%%%%%%%%%%%%%%%%%%%%%%
% Penalties
%%%%%%%%%%%%%%%%%%%%%%%%%%%%%%%%%%%%%%%%%%%%%%%%%%%%%%%%%%%%%%%%%%%%%%%%

\clubpenalty=10000
\displaywidowpenalty=10000
\widowpenalty=10000

\usepackage[defaultlines=100,all]{nowidow} % rejects all orphaned and widowed lines
\setlength\parindent{0pt}

%%%%%%%%%%%%%%%%%%%%%%%%%%%%%%%%%%%%%%%%%%%%%%%%%%%%%%%%%%%%%%%%%%%%%%%%
% Main document
%%%%%%%%%%%%%%%%%%%%%%%%%%%%%%%%%%%%%%%%%%%%%%%%%%%%%%%%%%%%%%%%%%%%%%%%
\begin{document}

{\huge \textsc{Abhiram Mullapudi}}\\
{\small \textit{Curriculum vitae} (\today)}\\


{\scriptsize
\faEnvelope\space{\href{mailto:abhiramm@umich.edu}{abhiramm@umich.edu \textsuperscript{\faExternalLink}}}
\faHome\space{\href{https://randomstorms.net}{randomstorms.net \textsuperscript{\faExternalLink}}}
\faGithub\space{\href{https://github.com/abhiramm7}{github/abhiramm7 \textsuperscript{\faExternalLink}}}
\faLeanpub\space{\href{https://scholar.google.com/citations?user=W9mUihUAAAAJ&hl=en}{google-scholor/abhiramm \textsuperscript{\faExternalLink}}}}


\section*{Positions}

\years{2023--}
\textbf{Senior Data Scientist} at Xylem
\begin{itemize}
	\item \textbf{Timeseries forecasting \& Anomaly Detection}: I develop algorithms for filtering timeseries data from sensors and detecting anamolous behaiours and forecasting 
	\item \textbf{MLOps}: I work on development of flyte-based mlops-platform for the developing and deployment machine-learning-based analytical services.
\item \end{itemize}

\years{2020--2023}
\textbf{Hydraulic Control and Optimization Engineer} at Xylem
\begin{itemize}
\item Timeseries forecasting: Developed a one-dimentional convoluttional neural network for predicting 24 hours flows at the treatmeant plant in near-real time. 
\item Timeseries filtering: Developed a one-dimentional convolution neural network model for filtering and interpolating timeseries spatio-temporally.
\item Simlation of stormwater systems: Developed an extension for EPA-SWMM to identify travel-times and extended EPA-SWMM to assimilate sensor data into simulation by dynamially updating swmm files.
\item Real-time applications:
- Developed an real-time timeseries processing module for filtering sensor data streams, applying aribary equations using symbolic programming, and  detect anamoloues behaviour in water networks using 
\end{itemize}


%%%%%%%%%%%%%%%%%%%%%%%%%%%%%%%%%%%%%%%%%%%%%%%%%%%%%%%%%%%%%%%%%%%%%%%%
\section*{Education}
%%%%%%%%%%%%%%%%%%%%%%%%%%%%%%%%%%%%%%%%%%%%%%%%%%%%%%%%%%%%%%%%%%%%%%%%

\years{2017--2020}%
\textbf{Ph.D.}\ in Civil Engineering at University of Michigan, Ann Arbor, USA\\
\emph{Statistical Learning Approaches for the Control of Stormwater Systems}; Advisor: Branko Kerkez \\[.1cm]

\years{2015--2017}%
\textbf{M.Sc.Eng.}\ in Civil Engineering at University of Michigan, Ann Arbor, USA\\[.1cm]

\years{2011--2015}%
\textbf{B.Tech}\ in Civil Engineering at Amrita Vishwa Vidyapeetham, Coimbatore, India\\[.1cm]

%%%%%%%%%%%%%%%%%%%%%%%%%%%%%%%%%%%%%%%%%%%%%%%%%%%%%%%%%%%%%%%%%%%%%%%%
\section*{Publications}
%%%%%%%%%%%%%%%%%%%%%%%%%%%%%%%%%%%%%%%%%%%%%%%%%%%%%%%%%%%%%%%%%%%%%%%%
\years{2023} Abhiram Mullapudi and Branko Kerkez. Bayesian optimization for shaping
the response of stromwater networks.

\years{2023} Sara P. Rimer, Abhiram Mullapudi, Sara C. Troutman, Gregory Ewing, Jef-
frey M. Sadler, Jonathan L. Goodall, Ruben Kertesz, Jon M. Hathaway, and
Branko Kerkez. pystorms: a simulation sandbox for the design and evaluation
of stormwater control algorithms. Environmental Modelling and Software,
2023

%%%%%%%%%%%%%%%%%%%%%%%%%%%%%%%%%%%%%%%%%%%%%%%%%%%%%%%%%%%%%%%%%%%%%%%%
\section*{Honours}
%%%%%%%%%%%%%%%%%%%%%%%%%%%%%%%%%%%%%%%%%%%%%%%%%%%%%%%%%%%%%%%%%%%%%%%%

\years{2018}
Winner of LIFT challange 

\section*{Skills}

\section*{Professional Service}
\begin{itemize}
	\item Vice-chair of Emerging and Innovative Technologies subcommittee for American Socity of Civil Engneering's Environmental and Water Research Congress. 
	\item Part of the organizing committee at 2024 NURAL-IPS Gaussian Processes workshop
\end{itemize}
\subsection*{Journal Review}
\begin{itemize}
	\item \textit{HardwareX}
	\item \textit{IEEE-CDC 2020}
	\item \textit{Journal of Hydrology}
	\item \textit{Water Resources Research}
	\item \textit{Journal of Hydroinformatics}
	\item \textit{Water Science and Technology}
	\item \textit{Journal of Open Source Software}
	\item \textit{Journal of Computing in Civil Engineering}
	\item \textit{Journal of Irrigation and Drainage Engineering}
	\item \textit{Journal of Water Resources Planning and Management}
	\item \textit{Environmental Science: Water Research \& Technology}
\end{itemize}

\section*{Conference}


%%%%%%%%%%%%%%%%%%%%%%%%%%%%%%%%%%%%%%%%%%%%%%%%%%%%%%%%%%%%%%%%%%%%%%%%
\section*{About this template}
%%%%%%%%%%%%%%%%%%%%%%%%%%%%%%%%%%%%%%%%%%%%%%%%%%%%%%%%%%%%%%%%%%%%%%%%

This document is modelled after the style of
\href{https://bastian.rieck.me}{my} own CV, which you can find
\href{https://bastian.rieck.me/about/cv.pdf}{here}. It was originally
inspired by the CV of \href{http://nitens.org/taraborelli/cvtex}{Dario
Taraborelli}, but has since somewhat evolved---or so I like to tell
myself---from the original template.\\

This document does not offer many special features, except for the
\verb|\years| macro, which can be used to typeset small notes in the
margin of the document. I use them to indicate \emph{durations}, but you
could also repurpose it to have small annotations in the style of Edward
Tufte. In addition, the template uses two special fonts for sans-serif
typesetting and monospace typesetting. I do not tend to use the former
for many things, but you might like it for typesetting the titles of
publications. The latter font type, though, I often use in order to
describe software projects or packages, such as \texttt{PyTorch} or
\texttt{scikit-learn}.\\

That's all there is to it---enjoy the template \& feel free to open
tickets for comments or feedback.

\end{document}
