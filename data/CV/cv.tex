\documentclass{my_cv}

\begin{document}
% bibliography
\nobibliography{mypub}
\bibliographystyle{unsrt}
\thispagestyle{plain}
\name{\textbf{Abhiram Mullapudi}}{randomstorms.net}{abhiramm@umich.edu}


\section*{EDUCATION}

\datedsubsection{\textbf{Ph.D in Civil Engineering} (Intelligent Systems)}{2020}{\vspace*{-0.25cm}\textit{University of Michigan, Ann Arbor, USA}}\\
Dissertation: Statistical Learning Approaches for the Control of Stormwater Systems\\
Advisor: Dr.Branko Kerkez
\datedsubsection{\textbf{M.Sc.Eng.\ in Civil Engineering} (Intelligent Systems)}{2017}{\vspace*{-0.25cm}\textit{University of Michigan, Ann Arbor, USA}}
\datedsubsection{\textbf{B.Tech. (distinction) in Civil Engineering}}{2015}{\vspace*{-0.25cm}\textit{Amrita Vishwa Vidhyapeetham, Coimbatore, India}}

\section*{EXPERIENCE}

\datedsubsection{\textbf{Hydraulic Control and Optimization Engineer}}{2020-}{\vspace*{-0.25cm}\textit{Xylem Inc.}}\\
\noindent Developing optimization and control strategies for the effective management of urban water systems.

\datedsubsection{\textbf{Graduate Student Research Assistant}}{2016--2020}{\vspace*{-0.25cm}\textit{Real-time Water Systems Lab, University of Michigan}}\\
\noindent Development of algorithms, simulation tools, and open source hardware solutions for monitoring and control of stormwater networks.

\datedsubsection{\textbf{Research Assistant}}{2015}{\vspace*{-0.25cm}\textit{Love Biotechnology Group, University of Michigan}}\\
\noindent Characterization of influent and calibration of process model for Detroit's waste water treatment plant. 

\datedsubsection{\textbf{Research Assistant}}{2014}{\vspace*{-0.25cm}\textit{Department of Chemical Engineering, Amrita Vishwa Vidhyapeetham}}\\
\noindent Aided in the design, construction, and monitoring of a vertical flow constructed wetland. 

\section*{AWARDS}
\datedsubsection{Grand prize winner, LIFT Intelligent Water Systems Challenge}{2018}{}
\datedsubsection{Academic Excellence, Amrita Vishwa Vidhyapeetham}{2013, 2015}{}

\section*{PUBLICATIONS}	
\begin{enumerate}
	\item \bibentry{Rimer2019bench}\ (\textit{in preparation, \href{https://randomstorms.net/data/papers/pystorms_exabs.pdf}{extended abstract}})
	\item \bibentry{Mullapudi2019bay}\ (\textit{in preparation, \href{https://randomstorms.net/data/papers/bayes_poster.pdf}{poster}})
	\item \bibentry{mullapudi2020statistical}
	\item \bibentry{Mullapudi2020pyswmm}
	\item \bibentry{Mullapudi2020rl}
	\item \bibentry{bartos2019rrcf} 
	\item \bibentry{Mullapudi2018}
	\item \bibentry{Mullapudi2017} 
\end{enumerate}

\section*{CONFERENCES}
\begin{enumerate}
	\item \bibentry{2021ICWMM}
	\item \bibentry{2020Borch}
	\item \bibentry{2019AGUFM}
	\item \bibentry{2019UFKON}
	\item \bibentry{2019watermatex}
	\item \bibentry{2019ccwi}
	\item \bibentry{2019ewri}
	\item \bibentry{water[at]mich}
	\item \bibentry{Rimer_2019}
	\item \bibentry{2019midce}
	\item \bibentry{2018weftec}
	\item \bibentry{2018hic}
	\item \bibentry{2018hicl}
	\item \bibentry{2018ewri}
	\item \bibentry{2018ewril}
	\item \bibentry{2017AGUFM}
	\item \bibentry{2017AEESP}
	\item \bibentry{2017ica}
	\item \bibentry{2017rldm}
	\item \bibentry{2017ewri}
	\item \bibentry{2017ewrip}
	\item \bibentry{2016AGUFM}
\end{enumerate}

\section*{PROGRAMMING AND SCIENTIFIC COMPUTING}

\begin{itemize}
	\item Proficient programmer in Python, MATLAB, C/C++, OpenMP, CUDA, \LaTeX, and bash.
	\item \textbf{Machine Learning Stack:} Experienced in using TensorFlow, PyTorch, GpyOpt for training large scale machine learning models in high performance clusters. 
	\item \textbf{Embedded Systems:} Developer of Open-Storm's \href{https://github.com/open-storm/perfect-cell}{\texttt{perfect-cell}}, an open source operating system for environmental monitoring. Experienced in using EAGLE and Cypress modules for designing customized hardware.  
	\item \textbf{Cloud Computing:} Experienced in using cloud computing services (AWS, Google cloud, and Azure) for creating backends for streaming data from IoT devices. 
	\item \textbf{Stormwater Modeling:} Creator of \href{https://klabum.github.io/pystorms/}{\texttt{pystorms}}, an open source python library for the design and evaluation of stormwater control algorithms. Contributor to Open Water Analytics's \href{https://github.com/OpenWaterAnalytics/Stormwater-Management-Model}{SWMM} and \href{https://github.com/OpenWaterAnalytics/pyswmm}{pyswmm}, the industry standard for modeling stormwater systems. 
	\item Contributor to \href{https://github.com/kLabUM/rrcf}{\texttt{rrcf}}, an open source implementation of an unsupervised learning algorithm for anomaly detection in live streaming data. 
\end{itemize}


\section*{WORKSHOPS}
%\datedsubsection{CPS-IoT Week - Smart Water, Smart Cities}{2020}{}
\datedsubsection{CUAHSI Open Source Urban Hydrology Sensor Bootcamp}{2017, 2019}{}

Co-organized and led a three day workshop on the use open-storm's sensing stack for the monitoring and control of stormwater systems.


\section*{PROFESSIONAL ACTIVITIES}

\begin{itemize}
	\item Peer reviewer for \textit{Journal of Hydrology}, \textit{Water Science and Technology}, \textit{Environmental Science: Water Research \& Technology}, \textit{Journal of Hydroinformatics}, \textit{Journal of Computing in Civil Engineering}, and \textit{IEEE-CDC 2020}. 
	\item Member of International Water Association's working group on real time control of urban drainage systems.
\end{itemize}

\section*{MEDIA COVERAGE}
\begin{itemize}
	\item 2018 LIFT Challenge: \href{https://cee.engin.umich.edu/stories/joint-u-m-and-glwa-team-wins-inaugural-intelligent-water-challenge/}{Grand Prize Winner}
	\item NSF Science Nation: \href{https://www.youtube.com/watch?v=mStPThxAU08}{Smart stormwater solutions for aging infrastructure}
\end{itemize}

\section*{References}
Available on request. 


\end{document}




